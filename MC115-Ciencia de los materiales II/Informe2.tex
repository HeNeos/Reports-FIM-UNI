\documentclass[a4paper,12pt]{report}
\usepackage[spanish,mexico]{babel}
\usepackage[utf8]{inputenc}
\usepackage[T1]{fontenc}
\usepackage{amsmath}
\usepackage{amssymb}
\usepackage{multirow}
\usepackage{wasysym}
\usepackage[dvipsnames,pdftex]{color}
\usepackage[colorinlistoftodos]{todonotes}
\usepackage[left=3.0cm,right=2.0cm,top=3.0cm,bottom=2cm]{geometry}
%\usepackage{helvet}
%\renewcommand{\familydefault}{\sfdefault}
\setlength{\oddsidemargin}{0in}
\usepackage{float}
 \setlength{\textwidth}{6in}
 \setlength{\topmargin}{0in}
 \setlength{\voffset}{-0.5in}
 \setlength{\hoffset}{0.5in}
 \setlength{\textheight}{9in}
 \setlength{\textwidth}{6in}
 \setlength{\topskip}{0in}
 \setlength{\parskip}{2ex}
 \renewcommand{\baselinestretch}{1}
\usepackage{diagbox}
\usepackage{array}
\usepackage{listings}
\usepackage{caption}
%%% comandos definidos por el usuario

\begin{document}
\setcounter{page}{1}
\pagenumbering{roman}
\thispagestyle{empty}
\begin{center}
{\huge UNIVERSIDAD NACIONAL DE INGENIERÍA}\\[0.9cm]
{\Large FACULTAD DE INGENIERÍA MECÁNICA}\\[0.6in]
\end{center}
\begin{figure}[h]
\begin{center}
\includegraphics[scale=0.2]{LogoUNI}
\vspace{0cm}
\end{center}
\end{figure}
\vspace{0.5cm}
\begin{center}
SEGUNDO INFORME DE LABORATORIO\\
CIENCIA DE LOS MATERIALES II\\[14mm]
{\Large RECOCIDO CONTRA ACRITUD}\\[10mm]
\vfill
LIMA - PERÚ \hfill OCTUBRE 2018
\end{center}
\newpage
\thispagestyle{empty}
\begin{center}
{\huge RECOCIDO CONTRA ACRTIDUD}\\[0.7cm]
\small ENTREGADO:\\[0.3cm]
\small 13 OCTUBRE 2018\\[0.9cm]
\end{center}
\begin{flushleft}
{\large INTEGRANTES:}\\[3cm]
\end{flushleft}
%\begin{tabular}{c@{\hspace{0.5in}}c}
%\rule[1pt]{2.6in}{1pt}&\rule[1pt]{2.6in}{1pt}\\
%Huaroto Villavicencio Josué, 20174070I & Fuentes Valdivia Martin, 2017\\[1.5cm]
%\rule[1pt]{2.6in}{1pt}&\rule[1pt]{2.6in}{1pt}\\
%Saldivar Montero Perruardo, 2017 & Nombre 3, 2017\\[1cm]
%\rule[1pt]{2.6in}{1pt}&\rule[1pt]{2.6in}{1pt}\\
%Nombre 4, 2017 & Nombre 5, 2017 \\[1.5cm]
\begin{tabular}{c@{\hspace{0.5in}}c}
\rule[1pt]{2.6in}{1pt}&\rule[1pt]{2.6in}{1pt}\\
Huaroto Villavicencio Josué, 20174070I & Landeo Sosa Bruno, 20172024J\\[2.5cm]
\rule[1pt]{2.6in}{1pt}&\rule[1pt]{2.6in}{1pt}\\
Quesquen Vitor Angel, 20170270C & Saldivar Montero Eduardo, 20174013E\\[2.5cm]
\rule[1pt]{2.6in}{1pt}&\rule[1pt]{2.6in}{1pt}\\
Saravia Echevarria Henrry, 20170233K & Sotelo Cavero Sergio, 20172125K \\[1.6cm]
\end{tabular}
%\\[0.7cm]
{\large PROFESOR:} \\[1.15cm]
\begin{center}
\begin{tabular}{c}
\rule[3pt]{4.8in}{1pt}\\[1pt]
ING. LUIS SOSA, JOSE 
\end{tabular}
\end{center}
\vfill
%\newpage
%\begin{center}
%{\Large \bf{RESUMEN}}
%\end{center}
\newpage
\tableofcontents
\listoffigures
\addcontentsline{toc}{chapter}{Índice de Figuras}
\chapter{Objetivos}
\begin{enumerate}
\item Conocer los fenómenos que ocurren cuando un material con deformación plástica en frío es sometido a un tratamiento térmico de recocido a fin de recuperar la estructura inicial.
\item Comparar el tamaño de grano después del recocido con el inicial, después de la deformación plástica asociada.
\item Relacionar la dureza final con la deformación plástica previa. 
\item Comparar la dureza final con la inicial obtenida en el anterior ensayo y así caracterizar el efecto del temple en función de la deformación plástica aplicada previamente. 
\end{enumerate}
\pagenumbering{arabic} %%% esto es para regresar el modo de numeración a numeración arábiga
\setcounter{page}{1}  %%% empezamos en página 1
\chapter{Recocido contra acritud}
\section{Materiales}
\begin{enumerate}
\item Lijas
\begin{figure}[H]
\begin{center}
\includegraphics[scale=0.32]{lija1500}
\end{center}
\caption{Lija 1500}
\end{figure}
\item Pulidora
\begin{figure}[H]
\begin{center}
\includegraphics[scale=0.2]{pulidora}
\end{center}
\caption{Pulidora con óxido de aluminio}
\end{figure}
\item Nital
\begin{figure}[H]
\begin{center}
\includegraphics[scale=0.2]{nital}
\end{center}
\caption{Frasco con nital}
\end{figure}
\item Microscopio
\begin{figure}[H]
\begin{center}
\includegraphics[scale=0.3]{microscopio}
\end{center}
\caption{Microscopio metalográfico}
\end{figure}
\end{enumerate}
\chapter{Procedimiento}
\begin{enumerate}
\item Las probetas que se ha utilizado en el Laboratorio de Deformación Plástica en frío, se someten a un recocido contra acritud a una temperatura de 820$^{\circ}$C por un tiempo de 40 minutos.
\item Después del paso anterior, se comienza con el desbaste de la probeta, para ello se utilizará primero la lija número 80 y se inicia a lijar de tal forma que, transcurrido cierto tiempo lijando en una dirección, se cambia una dirección perpendicular a la anterior, con el fin de eliminar la capa de rayado; cabe resaltar que durante este proceso se usa el agua constantemente, dado que ayuda a eliminar las impurezas.
\begin{figure}[H]
\begin{center}
\includegraphics[scale=0.35]{lijar}
\end{center}
\caption{Proceso de desbaste}
\end{figure}
\item Se repite el paso anterior con las lijas número 150, 320, 400, 600, 1200 en ese orden.
\newpage
\item Después del lijado, se procede con el pulido y secado.
\begin{figure}[H]
\begin{center}
\includegraphics[scale=0.16]{procesopulidora}
\end{center}
\caption{Pulidora en funcionamiento}
\end{figure}
\item Pulida y secada, la probeta se limpia y se somete al ataque químico del nital, después se le sumerge en alcohol para frenar la reacción y se lleva al microscopio.
\item Posteriormente, se le somete a un ensayo de dureza en durómetro Rockwell.
\begin{figure}[H]
\begin{center}
\includegraphics[scale=0.25]{mrod}
\end{center}
\caption{Durómetro Rockwell digital}
\end{figure}
\end{enumerate}
\chapter{Cálculos y resultados}
\section{Imágenes de la metalografía}
\subsubsection{Probeta 0}
\begin{figure}[H]
\begin{center}
\includegraphics[scale=0.4]{probeta0}
\end{center}
\end{figure}
\subsubsection{Probeta 1}
\begin{figure}[H]
\begin{center}
\includegraphics[scale=0.25]{probeta1}
\end{center}
\end{figure}
\subsubsection{Probeta 2}
\begin{figure}[H]
\begin{center}
\includegraphics[scale=0.25]{probeta2}
\end{center}
\end{figure}
\subsubsection{Probeta 3}
\begin{figure}[H]
\begin{center}
\includegraphics[scale=0.25]{probeta3}
\end{center}
\end{figure}
\subsubsection{Probeta 4}
\begin{figure}[H]
\begin{center}
\includegraphics[scale=0.35]{probeta4}
\end{center}
\end{figure}
\subsubsection{Probeta 5}
\begin{figure}[H]
\begin{center}
\includegraphics[scale=0.25]{probeta5}
\end{center}
\end{figure}
\section{Tamaño de grano}
Utilizaremos la fórmula:
$$
\left(\frac{A}{100}\right)^{2}\cdot\frac{\#granos}{pulg^{2}}=2^{n-1}
$$
\subsubsection{Probeta 0}
\hspace{-19pt}Aumento: $\times$100\\
$\#$granos: 8+$\frac{14}{2}$\\
n=4.901$\approx$5
\subsubsection{Probeta 1}
\hspace{-19pt}Aumento: $\times$100\\
$\#$granos: 10+$\frac{14}{2}$\\
n=5.080$\approx$5
\subsubsection{Probeta 2}
\hspace{-19pt}Aumento: $\times$100\\
$\#$granos: 9+$\frac{44}{2}$\\
n=4.301$\approx$4
\subsubsection{Probeta 3}
\hspace{-19pt}Aumento: $\times$100\\
$\#$granos: 8+$\frac{16}{2}$\\
n=5
\subsubsection{Probeta 4}
\hspace{-19pt}Aumento: $\times$100\\
$\#$granos: 8+$\frac{9}{2}$\\
n=4.640$\approx$4
\subsubsection{Probeta 5}
\hspace{-19pt}Aumento: $\times$100\\
$\#$granos: 10+$\frac{16}{2}$\\
n=5.160$\approx$5
\section{Dureza después de recocido y \% Deformación inicial}
\begin{center}
\begin{tabular}{|c|c|c|}
\hline 
Probeta & Dureza & $\%$ Deformación \\ 
\hline 
Probeta 0 & 56.1 - 58.0 & 0 \\ 
\hline 
Probeta 1 & 61.4 - 56.2 & 18.1213 \\ 
\hline 
Probeta 2 & 60.0 - 59.0 & 27.6024 \\ 
\hline 
Probeta 3 & 57.7 - 58.7 & 33.4221 \\ 
\hline 
Probeta 4 & 55.2 - 59.3 & 46.8165 \\ 
\hline 
Probeta 5 & 70.7 - 70.6 & 55.4728 \\ 
\hline 
\end{tabular} 
\end{center}
\section{Gráfico 1}
\begin{figure}[H]
\begin{center}
\includegraphics[scale=0.4]{grafico1}
\end{center}
\end{figure}
\section{Gráfico 2}
\begin{figure}[H]
\begin{center}
\includegraphics[scale=0.4]{grafico2}
\end{center}
\end{figure}
\section{Gráfico 3}
\begin{figure}[H]
\begin{center}
\includegraphics[scale=0.4]{grafico3}
\end{center}
\end{figure}
\chapter{Conclusiones y recomendaciones}
\begin{enumerate}
\item Los metales que han sido sometidos a deformación plástica adquieren una propiedad llamada acritud, que consiste %en un aumento de dureza, fragilidad y la resistencia a la deformación; sin embargo, se ve afectada la maleabilidad y %ductilidad. Para devolverle estas propiedades, se recurre al tratamiento térmico de recocido contra acritud.
\item De acuerdo a las microfotografías obtenidas antes y después del tratamiento, se concluye que los metales con deformación sin recocido poseen una estructura deformada, además de alta dureza y baja ductilidad. Cuando se los somete al recocido, estos recuperan la forma de su estructura (recristalizada), su dureza disminuye y también recupera su ductilidad.
\item Las probetas ensayadas, luego del recocido, resultaron con estructuras similares entre sí. Esto queda demostrado con los índices de grano parecidos, a pesar de tener distintos grados de deformación. Sin embargo, esto no es correcto, ya que mientras mayor es la deformación, el tamaño de grano debería ir disminuyendo.
\item Se comprueba la disminución de la dureza del material recocido respecto al únicamente deformado, lo cual confirma que las tensiones internas se eliminaron con el tratamiento.
\item Tener como mínimo 2 valores de las durezas de las probetas para un mejor estudio del cambio del mismo y demás propiedades del material (cobre).
\item No sobrepasarse el tiempo estimado para el recocido, el cuál en nuestro caso fue entre un aproximado de 20 a 30 minutos, esto provocaría que el objetivo principal de devolver las propiedades mecánicas del material se vean afectadas.
\item Se debe procurar que las superficies de las probetas estén bien lijadas, pulidas y con las caras planas antes de realizar el ensayo metalográfico, ya que en las microfotografías no se podrán apreciar con claridad la forma y tamaño de grano, y esto afectaría a los cálculos. 
\end{enumerate}
\chapter{Anexos}
\section{Cuestionario}
\begin{enumerate}
\item \textbf{Mencionar las etapas que ocurren durante el recocido contra acritud de un metal que ha sido deformado en frío.}\\
\begin{itemize}
\item \textbf{Restauración o recuperación.}\\
Se eliminan las deformaciones de la red cristalina, no requiere una temperatura alta ya que se produce una traslación insignificante de los átomos. En el caso del hierro es un pequeño calentamiento entre (300-400$^{\circ}$C). Se elimina la deformación de la red como resultado de numerosos submicroprocesos: disminución de la densidad de las dislocaciones a consecuencia de su aniquilación mutua, fusión de los bloques, disminución de las tensiones internas, reducción del número de huecos, etc. En esta etapa la dureza disminuye de un 20 a 30\%.
\item \textbf{Recristalización.}
\begin{itemize}
\item \textbf{Recristalización primaria o del tratamiento,} durante la cual los granos alargados a consecuencia de la deformación plástica se transforman en granos pequeños redondeados orientados irregularmente. Se forman nuevos granos.
\item \textbf{Recristalización secundaria o colectiva,} que consiste en el crecimiento de los granos y que transcurre a una temperatura más elevada.
\end{itemize}
En la recristalización se forman nuevos granos, se realiza a temperaturas más altas y puede comenzar con una velocidad apreciable después de calentar el metal por encima de una temperatura determinada. Cuanto mayor es la pureza del metal más baja es la temperatura de recristalización determinada. Una vez terminada la recristalización, la estructura del material y sus propiedades vuelven a ser las de antes (las que tenía antes de ser deformado). 
\end{itemize}
\item \textbf{Explica por que cambia la resistencia eléctrica de un material metálico durante un recocido contra acritud. ¿En qué etapa del recocido contra acritud se produce el mayor cambio de esta resistencia?}\\
Cambia porque se desconcentran defectos puntuales y aumenta la movilidad atómica ya que la resistividad aumenta con la densidad de defectos de la red, vacantes, átomos intersticiales, que provocan la resistencia al flujo electrónico por aumento de choques, entre los electrones y átomos. Esto se produce en la etapa de restauración o recuperación, ya que las dislocaciones se mueven a zonas de más baja energía.
\item \textbf{Durante el recocido contra acritud de un metal deformado en frío ¿Cómo ocurre el crecimiento de los granos?}\\
En el crecimiento o coalescencia de los granos (recristalización secundaria), la temperatura hace que aumenten de tamaño algunos granos a expensas de otros pequeños, si la temperatura sigue elevándose aumenta el número de granos grandes hasta que todos granos pequeños son absorbidos por los grandes y toda la estructura queda formada por granos grandes. Este proceso es espontáneo por la tendencia del sistema a disminuir la reserva de energía interna pues al aumentar de tamaño los granos se ocasionará una disminución en general de las superficies en contacto y, por consiguiente, una disminución de la reserva de la energía interna del sistema. Existen tres mecanismos posibles de crecimiento de grano:
\begin{itemize}
\item El embrionario, que consiste en que después de la recristalización primaria aparecen otra vez los centros embrionarios de cristales nuevos y su crecimiento hacen que se formen nuevos granos, pero su numero es menos que los granos del estado inicial y por esto cuando termina el proceso, aquellos son por término medio más grandes.
\item El migratorio, consiste en la traslación de los límites de grano y el aumento de sus dimensiones. Como el grano grande es termodinámicamente más estable que el pequeño, porque la relación $(S/V)$ es más pequeña para él, siendo $S$ la superficie y $V$ el volumen. Los granos grandes crecen a costa de comerse a los granos pequeños. 
\item El de fusión de los granos, en el cual los límites de los granos van “desapareciendo” poco a poco y muchos granos pequeños se unen formando uno grande.
\end{itemize}
\item \textbf{Se tienen dos probetas de hierro electrolítico deformadas 5\% y 50\%, que son recocidas a 800 grados centígrados, ¿Cuál de ellas alcanzará un grano más grande después del mismo tiempo de permanencia en el horno a alta temperatura?}\\
Podemos de decir que la probeta de menor deformación (5\%) alcanzará un grano más grande con respecto del que tiene mayor deformación (50\%).\\
Decimos esto porque existe un grado de deformación que condiciona principalmente el desarrollo del proceso de fusión y que conduce después del calentamiento a un crecimiento gigantesco de los granos; se llama grado de deformación crítico, este grado no es muy grande y se halla entre los limites del 3\% al 8\% (generalmente). Si sobrepasa este limite el crecimiento del grano se produce como resultado de la migración de los límites, lo que a igualdad de las demás condiciones da un grano más fino que el que se obtiene por el proceso de fusión.
\begin{figure}[H]
\centering
\includegraphics[scale=0.55]{preg4}
\caption{Tamaño de grano (recristalización) VS grado de deformación}
\end{figure}
\item \textbf{¿A qué se denomina textura de un material deformado en frío?}\\
Se le denomina textura a la forma y orientación en la que quedan los granos después de ser sometidos a la deformación en frío, es decir, a lo largo del proceso, los granos giran y al mismo tiempo se alargan, haciendo que ciertas direcciones y planos cristalográficos queden alineados; en consecuencia, se desarrollan orientaciones o texturas (microestructuras fibrosas). 
\begin{figure}[H]
\centering
\includegraphics[scale=0.4]{preg5}
\caption[Estructura granular fibrosa de un acero de bajo carbono]{Estructura granular fibrosa de un acero de bajo carbono.}{Trabajo en frío a 60\% y 90\% de izquierda a derecha}
\end{figure}
\item \textbf{¿En qué se diferencia el proceso de recristalización para un material deformado en frío de uno deformado en caliente?}\\
La deformación en frío se realiza a una temperatura menor a la de recristalización y recién este proceso se lleva a cabo en el recocido, donde elimina la mayoría de los efectos del endurecimiento por deformación y disminuye significativamente las dislocaciones. En cambio, la deformación en caliente se realiza por encima de la temperatura de recristalización y además durante el trabajo en caliente el metal se estará cristalizando continuamente, en consecuencia, no necesitará del recocido.
\begin{figure}[H]
\centering
\includegraphics[scale=0.55]{preg6}
\end{figure}
\end{enumerate}
\begin{thebibliography}{99}  %%%este es un contador para el número de bibliografías utilizados.
\addcontentsline{toc}{chapter}{Bibliograf\'{\i}a} %%% Para introducir la bibliografía en el índice.
\bibitem{Keyser}{Keyser, Carl. ``Técnicas de Laboratorio para prueba de Materiales''. {\em{Limusa-Wiley.}}}
\bibitem{Zolotorevski}{Zolotorevski, V. ``Pruebas Mecánicas y Propiedades de los Metales''.{\em{Editorial MIR.}}}
\bibitem{Lasheras}{Lasheras. ``Tecnología de los Materiales Industriales''.} 
%\bibitem{Dieter}{Dieter. ``Metalurgia mecánica''.}
\bibitem{Apraiz}{Apraiz, J. ``Tratamiento Térmico de los Aceros''.}
\bibitem{Smith}{Smith, William F. y Ph.D. Hashemi, Javad ``Ciencia e ingeniería de materiales". {\em{
Madrid: McGraw-Hill, Interamericana de España.}} 570, (2004).} 
\bibitem{Callister}{Callister, William D. y Rethwisch, David G. ``Introducción a la ingeniería de los materiales''. {\em{Barcelona Reverté.}}, 960, (2007).} 
\bibitem{Askeland}{Askeland, Donald R., Pradeep P. Phulé y Wright, Wendelin J. ``Ciencia e ingeniería de los materiales''.{\em{México, D.F. Internacional Thomson Editores.}} {\textit{$6^{ta}$ edición}}, 1004, (2012).}
\end{thebibliography}
\end{document}
