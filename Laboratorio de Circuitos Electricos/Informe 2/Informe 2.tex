\documentclass[a4paper,12pt]{report}
\usepackage[spanish,mexico]{babel}
\usepackage[utf8]{inputenc}
\usepackage[T1]{fontenc}
\usepackage{amsmath}
\usepackage{amssymb}
\usepackage{wasysym}
\usepackage[dvipsnames,pdftex]{color}
\usepackage[x11names]{xcolor}
\usepackage{tikz, tkz-euclide}
\usepackage[american]{circuitikz}
\usepackage{siunitx}
\usetikzlibrary{arrows}
\usepackage[colorinlistoftodos]{todonotes}
%\usepackage[left=2cm,right=1.5cm,top=1cm,bottom=1cm]{geometry}
%\usepackage{helvet}
%\renewcommand{\familydefault}{\sfdefault}
\setlength{\oddsidemargin}{0in}
\usepackage{geometry}
\geometry{a4paper, total = {180mm,270mm},
			left = 25mm, top = 20mm,
            right=15mm,bottom=20mm,%
            footskip=10mm}
\usepackage{float} 
% \setlength{\topmargin}{0in}
% \setlength{\voffset}{-0.5in}
% \setlength{\hoffset}{0.3in}
% \setlength{\textheight}{700pt}
% \setlength{\textwidth}{440pt}
% \setlength{\topskip}{0in}
% \setlength{\parskip}{2ex}
 \renewcommand{\baselinestretch}{1.5}
\usepackage{diagbox}
\usepackage{array}
\usepackage{listings}
\usepackage{caption}
%%% comandos definidos por el usuario
\begin{document}
\setcounter{page}{1}
\pagenumbering{roman}
\thispagestyle{empty}
\begin{center}
{\huge UNIVERSIDAD NACIONAL DE INGENIERÍA}\\[0.9cm]
{\Large FACULTAD DE INGENIERÍA MECÁNICA}\\[0.6in]
\end{center}
\begin{figure}[h]
\begin{center}
\includegraphics[scale=0.33]{logoUNI.png}
\vspace{0cm}
\end{center}
\end{figure}
\vspace{0.5cm}
\begin{center}
INFORME DE LABORATORIO\\
LABORATORIO DE CIRCUITOS ELÉCTRICOS\\[5mm]
{\large TEOREMAS DE THEVENIN, NORTON Y MÁXIMA POTENCIA DE TRANSFERENCIA}\\[10mm]
\vfill
LIMA - PERÚ \hfill SEPTIEMBRE 2019
\end{center}
\newpage
\thispagestyle{empty}
\begin{center}
{\Huge TEOREMAS DE THEVENIN, NORTON Y MÁXIMA POTENCIA DE TRANSFERENCIA}\\[0.7cm]
\small ENTREGADO:\\[0.05cm]
\small 11 SEPTIEMBRE 2019\\[1.2cm]
\end{center}
\begin{flushleft}
{\large ALUMNO:}\\[2cm]
\end{flushleft}
%\begin{center}
%\begin{tabular}{c@{\hspace{0.5in}}c}
%\rule[1pt]{3.14in}{1pt}\\
%Sotelo Cavero Sergio, 20172125K% & Nombre 5, 2017 \\[1.5cm]
%\end{tabular}
%\end{center}
%\begin{center}
%\begin{tabular}{c@{\hspace{0.6in}}c}
%\rule[1pt]{3.14in}{1pt}\\
%Huaroto Villavicencio Josué, 20174070I \\[2cm]
%\rule[1pt]{3.14in}{1pt}\\
%Landeo Sosa Bruno, 20174070I \\[2cm]
%\rule[1pt]{3.14in}{1pt}\\
%Quesquén Vitor Angel, 20172125K \\[2cm]
%\rule[1pt]{3.14in}{1pt}\\
%Sotelo Cavero Sergio, 20172125K \\[2cm]
%\end{tabular}
%\end{center}
\begin{center}
\begin{tabular}{c}
\rule[1pt]{3.14in}{1pt}\\
Huaroto Villavicencio Josué, 20174070I \\[2.5cm]
\end{tabular}
\end{center}

%\rule[1pt]{3.14in}{1pt}\\
%Maguiña Amaya Wladimir, 20172019F \\[3cm]
%\rule[1pt]{3.14in}{1pt}\\
%Luis Sosa Jose, 19774147I \\[3cm]
%\rule[1pt]{3.14in}{1pt}\\
%Sotelo Cavero Sergio, 20172125K
%\end{tabular}
%\end{center}
%\\[0.7cm]
{\large PROFESOR:} \\[2cm]
\begin{center}
\begin{tabular}{c}
\rule[3pt]{4.8in}{1pt}\\[1pt]
ING. SINCHI YUPANQUI, FRANCISCO 
\end{tabular}
\end{center}
\vfill
%\newpage
%\begin{center}
%{\Large \bf{RESUMEN}}
%\end{center}
\newpage
\tableofcontents
%\listoffigures
%\addcontentsline{toc}{chapter}{Índice de figuras}
\newpage
\pagenumbering{arabic} %%% esto es para regresar el modo de numeración a numeración arábiga
\setcounter{page}{1}  %%% empezamos en página 1
%\part{Introducción}
\chapter{Objetivos}
\begin{enumerate}
\item Tomar en consideración las medidas de seguridad indicadas para la realización de un buen trabajo en el laboratorio.
\item Verificar experimentalmente los teoremas de Thevenin y Norton y  maxima potencia eléctrica.
\item Conocer mejor nuestro laboratorio de circuitos y sus alcances mediante esta experiencia.
\end{enumerate}
\chapter{Teorema de Thevenin}
El teorema de Thévenin establece que si una parte de un circuito eléctrico lineal está comprendida entre dos terminales A y B, esta parte en cuestión puede sustituirse por un circuito equivalente que esté constituido únicamente por un generador de tensión en serie con una resistencia, de forma que al conectar un elemento entre los dos terminales A y B, la tensión que queda en él y la intensidad que circula son las mismas tanto en el circuito real como en el equivalente.\\
El teorema de Thévenin fue enunciado por primera vez por el científico alemán Hermann von Helmholtz en el año 1853, pero fue redescubierto en 1883 por el ingeniero de telégrafos francés Léon Charles Thévenin, de quien toma su nombre.
\section{Cálculo de la tensión de Thevenin}
Para calcular la tensión de Thévenin, $V_{th}$, se desconecta la carga (es decir, la resistencia de la carga) y se calcula $V_{AB}$. Al desconectar la carga, la intensidad que atraviesa $R_{th}$ en el circuito equivalente es nula y por tanto la tensión de $R_{th}$ también es nula, por lo que ahora $V_{AB} = V_{th}$ por la segunda ley de Kirchhoff.
Debido a que la tensión de Thévenin se define como la tensión que aparece entre los terminales de la carga cuando se desconecta la resistencia de la carga también se puede denominar tensión en circuito abierto.\\
Para calcular la resistencia de Thévenin, se desconecta la resistencia de carga, se cortocircuitan las fuentes de tensión y se abren las fuentes de corriente. Se calcula la resistencia que se ve desde los terminales AB y esa resistencia $R_{AB}$ es la resistencia de Thevenin buscada $R_{th} = R_{AB}$.
\chapter{Teorema de Norton}
El teorema de Norton para circuitos eléctricos es dual del teorema de Thévenin. Se conoce así en honor al ingeniero Edward Lawry Norton, de los Laboratorios Bell, que lo publicó en un informe interno en el año 1926.​\\
Establece que cualquier circuito lineal se puede sustituir por una fuente equivalente de corriente en paralelo con una impedancia equivalente. Al sustituir un generador de corriente por uno de tensión, el borne positivo del generador de tensión deberá coincidir con el borne positivo del generador de corriente y viceversa. 
\section{Cálculo del Norton Equivalente}
El circuito Norton equivalente consiste en una fuente de corriente INo en paralelo con una resistencia RNo. Para calcularlo:
\begin{enumerate}
\item Se calcula la corriente de salida, IAB, cuando se cortocircuita la salida, es decir, cuando se pone una carga (tensión) nula entre A y B. Al colocar un cortocircuito entre A y B toda la intensidad INo circula por la rama AB, por lo que ahora IAB es igual a INo.
\item Se calcula la tensión de salida, $V_{AB}$, cuando no se conecta ninguna carga externa, es decir, cuando se pone una resistencia infinita entre A y B. $R_{No}$ es ahora igual a $V_{AB}$ dividido entre $I_{No}$ porque toda la intensidad $I_{No}$ ahora circula a través de $R_{No}$ y las tensiones de ambas ramas tienen que coincidir ( $VAB = I_{No}R_{No}$).
\end{enumerate}
\chapter{Teorema de máxima potencia}
El teorema establece cómo escoger (para maximizar la transferencia de potencia) la resistencia de carga, una vez que la resistencia de fuente ha sido fijada, no lo contrario. No dice cómo escoger la resistencia de fuente, una vez que la resistencia de carga ha sido fijada. Dada una cierta resistencia de carga, la resistencia de fuente que maximiza la transferencia de potencia es siempre cero, independientemente del valor de la resistencia de carga.\\
Para alcanzar la máxima eficiencia, la resistencia de la fuente (sea una batería o un dínamo) debería hacerse lo más pequeña posible. Bajo la luz de este nuevo concepto, obtuvieron una eficiencia cercana al 90\% y probaron que el motor eléctrico era una alternativa práctica al motor térmico. En esas condiciones la potencia disipada en la carga es máxima y es igual a: 
$$
P_{max} = \frac{V^{2}}{4R_{g}}
$$
\begin{thebibliography}{99}  %%%este es un contador para el número de bibliografías utilizados.
\addcontentsline{toc}{chapter}{Bibliograf\'{\i}a} %%% Para introducir la bibliografía en el índice.
%\bibitem{Rahman}{Rahman,Aminur y Doe, Hidekazu; ``Ion transfer of tetraalkylammonium cations at an interface between 
%frozen aqueous solution and 1,2-dichloroethane".{\em{Journal of Electroanalytical Chemistry}} {\bfseries 424},159,(1997).}
\bibitem{Gro}{Boylestad, Robert M. ``Introducción al análisis de circuitos''. {\em{Pearson}}}
\bibitem{Gro}{Sadiku, Matthew N. ``Fundamemtos de circuitos eléctricos''. {\em{Mc Graw Hill}}}
%\bibitem{Ding}{Ding, Zhifeng. ``Spectroelectrochemistry and photoelectrochemistry of charge transfer at liquid/liquid
%interfaces". {\em {Tesis, EPFL,}}(1999).}
%\bibitem{AL}{Alonso, Jose M. \em{Técnicas de mecanizado 1}}
%\bibitem{AL}{Alonso, Jose M. ``Técnicas de mecanizado 1". {\em{Paraninfo}} {\bfseries España-Madrid}, 6-20, (2001).}
%\bibitem{Samec2}{Samec Z., Lhotsky A., Jänchenová H., y Marecek, V. ``Interfacial tension and impedance measurements
%of interfaces between two inmiscible electrolyte solutions". {\em{Journal of Electroanalytical Chemistry}} {\bfseries
%43}, 47, (2000).}
%\bibitem{Day}{Day R.A. y Underwood A.L. {\textit{Química Analítica Cuantitativa}},5ºed. Prentice-Hall, México, 1998. 45-48.}
%\bibitem{Keyser}{Farah Abud, Michel. ``Determinación de la vida útil en herramientales de corte endurecido por el proceso de borurización en pasta''. {\em{Instituto tecnológico y de estudios superiores de Monterrey}}}
%\bibitem{Zolotorevski}{Escalona, I. ``Máquinas: herramientas por arranque de viruta.''.{\em{El Cid Editor.}}}
%\bibitem{Lasheras}{Lasheras. ``Tecnología de los Materiales Industriales''.} 
%\bibitem{Dieter}{Dieter. ``Metalurgia mecánica''.}
%\bibitem{Apraiz}{Apraiz, J. ``Tratamiento Térmico de los Aceros''.}
%\bibitem{Smith}{Smith, William F. y Ph.D. Hashemi, Javad ``Ciencia e ingeniería de materiales". {\em{
%Madrid: McGraw-Hill, Interamericana de España.}} 570, (2004).} 
%\bibitem{Callister}{Callister, William D. y Rethwisch, David G. ``Introducción a la ingeniería de los materiales''. %{\em{Barcelona Reverté.}}, 960, (2007).} 
%\bibitem{Askeland}{Askeland, Donald R., Pradeep P. Phulé y Wright, Wendelin J. ``Ciencia e ingeniería de los materiales''.{\em{México, D.F. Internacional Thomson Editores.}} {\textit{$6^{ta}$ edición}}, 1004, (2012).}
%\bibitem{HARDBANDING}{Tabla de conversión de escala de durezas. \begin{verbatim}http://%hardbandingsolutions.com/postle_sp/hardness.php
%\end{verbatim}}
%\bibitem{HE}{Fresadora. \begin{verbatim} http://lizdenbow.blogspot.com/
%\end{verbatim}}
%\bibitem{ASTM}{Normas ASTM.}
%\bibitem{NTP}{Normas NTP.}
\end{thebibliography}
\end{document}
