\documentclass[a4paper,11pt]{article}
\usepackage[utf8]{inputenc}
\usepackage[spanish]{babel}
\usepackage[left=2cm, right=2cm, top=2cm, bottom=2cm]{geometry}
\usepackage{amsfonts,amsmath,amssymb,amsthm}
\usepackage{graphicx}
\usepackage{here}
\newcommand{\mbf}{\mathbf} 
\newcommand{\mrm}{\mathrm}
\title{Segunda práctica calificada: ML202B}
\author{Josue Huaroto Villavicencio\\
Código: 20174070I}
\begin{document}
\maketitle
\section*{Problema 1}
\subsubsection*{Parámetros de cortocircuito}
$$
\theta_{cc} = \arccos\left (\frac{P_{NCu}}{V_{cc}I_{1n}} \right) = 73.210627339^{\circ}
$$
\begin{align*}
R_{eq1} &= \frac{P_{cc}}{I_{1n}^{2}} = 2.5161961\\
Z_{eq1} &= \frac{V_{cc}}{I_{1n}} = 8.71095\\
X_{eq1} &= \sqrt{Z_{eq1}^{2} - R_{eq1}^{2}} = 8.339634
\end{align*}
\subsubsection*{Parámetros de vacío}
$$
\theta_{vc} = \arccos\left (\frac{P_{NFe}}{V_{1n}I_{0}} \right) = 77.3659^{\circ}
$$
\begin{align*}
    g_{1} &= \frac{P_{NFe}}{V_{1n}^{2}} = 2.71177686\times 10^{-5}\\
    y_{1} &= \frac{I_{0}}{V_{1n}} = 0.00012398182\\
    b_{1} &= \sqrt{y_{1}^{2} - g_{1}^{2}} = 0.00012098
\end{align*}
\subsubsection*{Regulación a plena carga}
\begin{align*}
    R_{eq2} &= \frac{R_{eq1}}{a^{2}} = 0.629049\\
    X_{eq2} &= \frac{X_{eq1}}{a^{2}} = 2.0849084\\
    r\% &= \frac{I_{2n}(R_{eq2}+X_{eq2})}{V_{2}} \times 100\% = 5.607350047\%\\
    \alpha_{\max} &= \sqrt{\frac{P_{NFe}}{PNCu}} = 0.6354889
\end{align*}
\section*{Problema 2}
\begin{figure}[H]
    \centering
    \includegraphics[scale = 0.4]{c1.jpg}
\end{figure}
\begin{align*}
I_{N1} &= \frac{P}{V_{1}} = 208.33\\
I_{N2} &= \frac{P}{V_{2}} = 20.8333\\
a &= \frac{V_{1}}{V_{2}} = 0.1\\
I_{C} &= I_{L} = I_{N1}+I_{N2} = 229.1667\\
a' &= \frac{V_{1}+V_{2}}{V_{1}} = 11
\end{align*}
\begin{figure}[H]
    \centering
    \includegraphics[scale = 0.45]{c2.jpg}
\end{figure}
\begin{align*}
    S_{A} &= V_{2}I_{L} = 550000\\
    W_{T} &= P\times \mrm{fdp} = 40000\\
    W_{A} &= S_{A}\times \mrm{fdp} = 440000\\
    \Delta W &= \Delta W_{0} + \Delta W_{cc} = 803\\
    \eta_{t} &= 1-\frac{\Delta W}{\Delta W + W_{T}} = 98.032007\%\\
    \eta_{a} &= 1-\frac{\Delta W}{\Delta W + W_{A}} = 99.8178\%
\end{align*}
La eficiencia del autotransformador es mayor, esto significa que la potencia activa:
$$
W{A} > W_{T}
$$
\section*{Problema 3}
\begin{itemize}
    \item En los transformadores, para qué o qué propósito tiene realizar las pruebas de cortocircuito y circuito abierto.
    \begin{itemize}
        \item \textbf{Prueba de vacío}
        \begin{itemize}
            \item Se alimenta al trafo con su tensión normal y frecuencia nominal por uno de sus lados, mientras el otro se mantiene en circuito abierto.
            \item Se recomienda realizar estas pruebas por el lado de bajo tensión:
            $$
            P_{fe} = P_{H} + P_{F}
            $$
        \end{itemize}
        \item \textbf{Prueba de cortocircuito}
        \begin{itemize}
            \item Se alimenta al trafo con $I_{N}$ o de plena carga por uno de sus lados mientras el otro está en cortocircuito.
            \item Se recomienda realizar la prueba por el lado de alta tensión.
            \item Con esta prueba se miden las pérdidas por efecto Joule de los devanado, llamados pérdidas en el cobre.
        \end{itemize}
    \end{itemize}
    \item Dar 4 ventajas teóricas, de los transformadores ideales respecto a los reales.
    \begin{itemize}
        \item Los ideales al tener menos pérdidas, se reducirá el consumo eléctrico, haciendo que sea más eficiente.
        \item Ayuda a proteger el medio ambiente al tener menos pérdidas.
        \item Como los devanados 1 y 2 tienen resistencia despreciable entonces habrán menos pérdidas por efecto Joule.
        \item Generan menos calor, aumentando su vida útil.
    \end{itemize}
    \item En la práctica o en el laboratorio, explique cómo se determina la relación de transformación $a = ?$\\
    Se realiza por inyección de baja tensión en transformadores desenergizados y la correspondiente medición de la tensión incluida en otro devanado; la prueba de relación se hace para cada fase para cada toma si el transformador tiene varios tomas para cambiar su relación de tensión.
    \item El transformador en lado del secundario o carga, diga que pasa en los siguientes casos
    \item Diseño de un transformador de potencia\\
    Los transformadores son máquinas eléctricas estáticas, ya que convierte energía eléctrica a energía eléctrica, solo varía según su fase y una corriente específica, según el principio por el cual se diseñan sus potencias.\\
    La capacidad de un transformador se mide de acuerdo con el producto de su voltaje y su corriente, por ello el resultado es VoltAmperios; en la práctica se usa KVA.
\end{itemize}
\section*{Problema 4}
\begin{figure}[H]
    \centering
    \includegraphics[scale=0.51]{c4.jpg}
\end{figure}
\begin{figure}[H]
    \centering
    \includegraphics[scale=0.15]{c3.jpg}
\end{figure}
Índice horario: 210$^{\circ}$\\
Se denota: Dy7
\end{document}
