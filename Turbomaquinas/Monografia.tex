\documentclass[a4paper,12pt]{article}
\usepackage[T1]{fontenc}
\usepackage[utf8]{inputenc}
\usepackage[spanish]{babel}
\usepackage[left=1.5cm,right=1.5cm,top=1.4cm,bottom=1.5cm]{geometry}
\usepackage{graphicx}
\newcommand{\mrm}{\mathrm}
\usepackage{amsmath}
\usepackage{amsfonts}
\usepackage{amssymb}
\usepackage{here}
\begin{document}
\subsection*{Problema 7.}
    \textbf{Por qué razones no se puede instalar la turbina a gas a 4 000 msnm?. Fundamente su respuesta.}\\[4pt]
La razón es que, en una turbina a gas, no puede elevar suficiente relación de compresión o temperatura de entrada para las etapas de expansión. Por lo tanto, tiene un ciclo de gas con eficiencia limitada (35\% como máximo).\\
A 4000 msnm, la presión atmosférica es considerablemente menor a la de la costa; por lo tanto, se requiere destinar una mayor cantidad de potencia en el acondicionamiento del aire; es decir, el compresor de la instalación necesita más potencia, lo cual disminuye la eficiencia de la turbina.
\subsection*{Problema 17.}
\textbf{En una bomba de flujo axial, el rotor tiene un diámetro exterior de 75 cm y un diámetro interior de 40 cm, si gira a 500 RPM. En el radio medio del álabe , el ángulo en la entrada es de 128$^{\circ}$ y el ángulo a la salida es 158$^{\circ}$. Dibuje los diagramas de velocidad correspondiente en entrada y salida, y estimar a partir de ellos (1) la altura que la bomba generaría, (2) la descarga o la tasa de flujo en L/s, (3) la potencia al eje de entrada necesaria para accionar la bomba, y (4) la velocidad específica de la bomba. Supongamos una eficiencia manométrica o hidráulica de 88\% y una eficiencia total o global de 81\%}\\[4pt]
Se asume que la velocidad de entrada en la bomba axial es vertical:
\begin{figure}[H]
    \centering
    \includegraphics[scale=1.55]{17_vel.pdf}
    \caption{Triángulo de velocidades}
\end{figure}
\begin{itemize}
    \item $\beta_{1} = 128^{\circ}$
    \item $\beta_{2} = 158^{\circ}$
    \item $N = 500$ RPM
    \item $D_{1} = 75$e-2 m
    \item $D_{2} = 40$e-2 m
\end{itemize}
Con los datos del problema, se calculan las velocidades:
\begin{align}
    u &= \frac{N\pi D}{60} = 15.0534648\,\mrm{m/s}\\[3pt]
    c_{1} &= u\tan(180^{\circ} -\beta_{1}) = 19.2675563\,\mrm{m/s} = c_{m}\\[3pt]
    w_{1} &= \sqrt{u^{2} + c_{1}^{2}} = 24.45088\,\mrm{m/s}\\[3pt]
    w_{2} &= \frac{c_{m}}{\sin (180^{\circ} - \beta_{2})} = 51.4341088\,\mrm{m/s}\\[4pt]
    c_{2} &= \sqrt{w_{2}^{2} + u^{2} - 2w_{2}u\cos(180^{\circ} - \beta_{2})} = 37.898664\,\mrm{m/s}\\[4pt]
    c_{2u} &= w_{2}\cos(22^{\circ}) - u = 32.6354105\,\mrm{m/s}
\end{align}
Mientras que, la altura útil, caudal, potencia y velocidad específica:
\begin{align}
    H_{\infty} &= \frac{c_{2u}u}{\mrm{g}} = 50.079103288\,\mrm{m} \longrightarrow H_{\mrm{util}} = 88\%H_{\infty} = 44.06961089\,\mrm{m}\\[4pt]
    Q &= c_{m}\frac{\pi (D_{1}^{2}-D_{2}^{2})}{4} = 6.09091309\,\mrm{m^{3}/s} = 6090.91309 \mrm{L/s}\\[4pt]
    P_{\mrm{eje}} &= \frac{1000 Q H_{\mrm{util}}}{102\times 0.81} = 3248.9006277 \,\mrm{kW} = 4360.138\,\mrm{HP}\\[4pt] 
    N_{q} &= \frac{N\sqrt{Q}}{H^{3/4}_{\mrm{util}}} = 72.1463
\end{align}
\subsection*{Problema 27.}
\textbf{Marque con V si es verdadera o con F si es falsa a las siguientes afirmaciones:}\\[4pt]
\begin{itemize}
    \item En una turbina Francis con tubo difusor, la presión manométrica a la salida del rodete es negativa. \textbf{(V)}.\\
    Aumenta la altura útil y mejora la eficiencia, además si el tubo es difusor mejora más
    \item Teóricamente la velocidad relativa en la cuchara de la turbina Pelton es constante. \textbf{(V)}.\\
    Teóricamente sí es constante pues $u_{1}=u_{2}$, entonces $w_{1}=w_{2}$ para que $H_{\mrm{estatica}} = 0$.
    \item La altura útil de un aerogenerador es $C^{2}$/2g, donde $C$ es la velocidad del viento en la entrada y despreciable la  velocidad del viento a la salida. \textbf{(V)}.\\
     Sí, la velocidad de salida es casi nula, y como sale a la misma presión que entra, la única energía transmitida es la cinética.
    \item La turbina Francis es lenta cuando el ángulo de la velocidad relativa es mayor a 90$^{\circ}$. \textbf{(V)}.\\
    Pues $C_{2u}>u_{2}$, entonces $\beta_{2}$ es obtuso y es Francis lenta.
    \item Se inyecta agua a la cámara de combustión de una turbina a gas para aumentar la potencia y la eficiencia. \textbf{(F)}.\\
     La potencia aumenta pero la eficiencia disminuye.
\end{itemize}
\newpage
\subsection*{Problema 37.}
\textbf{En la fig. se muestra una bomba, donde la presión atmosférica es 14.7 PSIA, su eficiencia 79\%, para los datos indicados y tomando las consideraciones que sean necesarias. Determinar: 
    \begin{itemize}
        \item El caudal en m$^{3}$/s 
        \item La altura útil en m de agua 
        \item La potencia hidráulica y la potencia al eje en kW 
        \item Haga un diagrama h-s  donde se muestre la altura útil.
    \end{itemize}
    }
\begin{figure}[H]
    \centering
    \includegraphics[scale=0.8]{37.pdf}
    \caption{Bomba hidráulica}
\end{figure}
\subsubsection*{Caso 1:}
De las ecuaciones de energía en (0) y (2), (1) y (2) y la condición de continuidad del flujo:
\begin{align}
    H_{B} + \frac{P_{0}}{\gamma} + \frac{v_{0}^{2}}{2\mrm{g}} + z_{0} &= \frac{P_{2}}{\gamma} + \frac{v^{2}_{2}}{2\mrm{g}} + z_{2}\\[4pt]
    H_{B} + \frac{P_{1}}{\gamma} + \frac{v^{2}_{1}}{2\mrm{g}} + z_{1} &= \frac{P_{2}}{\gamma} + \frac{v^{2}_{2}}{2\mrm{g}} + z_{2}\\[4pt]
    \frac{v_{1}}{4^2} &= \frac{v_{2}}{6^2}
\end{align}
\begin{figure}[H]
    \centering
    \includegraphics[scale=1.5]{HS.pdf}
    \caption{Diagrama h-s}
\end{figure}
Resolviendo el sistema de ecuaciones no lineales:
\begin{align}
    v_{1} &= 4.41082633\,\mrm{m/s} \\[3pt]
    v_{2} &= 9.924359\, \mrm{m/s} \\[3pt]
    H_{B} &= 34.43210516\,\mrm{m} \\[3pt]
    Q &= A_{1}v_{1} = 0.08046\,\mrm{m^{3}/s}\\[4pt]
    P_{H} &= \frac{\rho Q H_{B}}{102} = 27.160838804\,\mrm{kW}\\[4pt]
    P &= \frac{P_{H}}{\eta} = 34.38080861\,\mrm{kW}
\end{align}
\subsubsection*{Caso 2:}
\begin{equation}
A_{2} = 0.0081073\,\mrm{m^{2}}, \hspace{20pt} A_{1} = 0.01824147\,\mrm{m^{2}}
\end{equation}
A falta de datos, se asume velocidades iguales a las áreas:
\begin{equation}
    v_{1} = 0.81073\,\mrm{m/s},\hspace{20pt} v_{2} = 1.824147\,\mrm{m/s} \longrightarrow Q = v_{1}A_{1} = 0.014789\,\mrm{m^{3}/s}
\end{equation}
\begin{align}
    H_{B} &= \frac{P_{2}-P_{1}}{h} + \frac{v^{2}_{2}-v^{2}_{1}}{2\mrm{g}} + z_{2}-z_{1}\\[4pt]
    H_{B} &= \frac{49.5-13}{10^{3}\times (0.0254)^{2}\times 2.2} + \frac{v^{2}_{2}-v^{2}_{1}}{2\mrm{g}} + (16\times 12 - 8)\times(0.0254)\\[4pt]
    H_{B} &= 30.5250837\,\mrm{m}
\end{align}
\begin{align}
    P_{H} &= \frac{1000 Q H_{B}}{102} = 4.42581\,\mrm{kW} \\[3pt]
    P &= 5.6022912\,\mrm{kW}
\end{align}
\subsection*{Problema 47.}
\textbf{Una turbina hidráulica llamada 1 opera con un salto de 50 m y el caudal de 10 m$^{3}$/s, el modelo llamado 2 en semejanza funciona con un salto de 10 m y el caudal de 100 L/s. se pide:
\begin{itemize}
    \item ¿Cuál es la relación entre la velocidad de rotación del modelo con la velocidad de rotación del prototipo?
    \item ¿Cuál es la relación entre el diámetro del modelo con la del prototipo? 
    \item Si el modelo genera una potencia de 9 KW ¿Cuál es la potencia de la turbina real?
\end{itemize}
}\\[4pt]
Turbina 1 (Prototipo)
\begin{itemize}
    \item $H_{p} = 50\,\mrm{m}$
    \item $Q_{p} = 10\,\mrm{m^{3}/s}$
\end{itemize}
Turbina 2 (Modelo)
\begin{itemize}
    \item $H_{m} = 10\,\mrm{m}$
    \item $Q_{m} = 0.1\,\mrm{m^{3}/s}$
\end{itemize}
a)\\
        $$
        N_{q} = \frac{NQ^{1/2}}{H^{3/4}} \longrightarrow \frac{N_{P}\sqrt{10}}{50^{3/4}} = \frac{N_{M}\sqrt{0.1}}{10^{3/4}} \longrightarrow \frac{N_{M}}{N_{P}} = 2.991
        $$
b)\\
        $$
        \psi = \frac{gH}{(ND)^{2}} = \frac{g\times 50}{(N_{P}D_{P})^{2}} = \frac{g\times 10}{(N_{M}D_{M})^{2}} \longrightarrow \frac{D_{M}}{D_{P}} = \sqrt{\frac{1}{S}} \frac{N_{P}}{N_{M}} = 0.1495
        $$
c)\\
        $$
        P_{M} = 9\,\mrm{kW}, \hat{P} = \frac{P}{\rho N^{3} D^{5}} \longrightarrow \frac{P_{P}}{\rho N_{P}^{3} D_{P}^{5}} = \frac{9}{\rho N^{3}_{M}D^{5}_{M}} \longrightarrow P_{P} = 4503.885\,\mrm{kW} = 6037.38\,\mrm{HP}
        $$
\end{document}
